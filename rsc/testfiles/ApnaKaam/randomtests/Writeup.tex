\documentclass[a4paper, 10pt]{article}
\usepackage{comment} % enables the use of multi-line comments (\ifx \fi) 
\usepackage{lipsum} %This package just generates Lorem Ipsum filler text. 
\usepackage{fullpage} % changes the margin

\begin{document}
%Header-Make sure you update this information!!!!
\noindent
\large\textbf{AI Assignment 2} \hfill \textbf{Kabir Chhabra Shikhar Murty} \\
\large CSL  333 \hfill \textbf{ 2013CS50287 2013EE10462} 



\section{Approach}
The problem was solved by breaking it into 4 main Boolean expressions.


\subsection{Every Edge belongs to a subgraph}
This boolean expression states that if e[i][j] exists then there exists a subgraph k, such that both i and j belong to k.

\subsection{No subgraph is a subset of another subgraph}
This boolean expression states that for all subgraphs k and l (k!=l), there exists an element in k that doesn't exist in l.
% to comment sections out, use the command \ifx and \fi. Use this technique when writing your pre lab. For example, to comment something out I would do:
%  \ifx
%	\begin{itemize}
%		\item item1
%		\item item2
%	\end{itemize}	
%  \fi

\subsection{All subgraphs are totally connected}
This boolean expression states that if there doesnt exist an edge between nodes i and j, then there is no subgraph such that both i and j are present in it.


\subsection{No subgraph is empty}
This boolean expression states that there is no subgraph such that no node belongs to it.\\\\
The clause that every node belongs to a subgraph is already covered in these cases.

\section{Acknowledgement}
We would like to acknowledge Akshay Gupta, Shreyas Padhy and Haroun Habeeb for their help.



\end{document}